\documentclass[12pt, a4paper, oneside, justified]{article}
\usepackage{amsmath, amsthm, amssymb, graphicx}
\usepackage{geometry}
\usepackage{ragged2e}
\usepackage[bookmarks=true, colorlinks, citecolor=blue, linkcolor=black]{hyperref}

\usepackage{tikz}
\usepackage{pythonhighlight}
\usetikzlibrary{patterns}

\geometry{
    a4paper, % 页面尺寸
    left=3cm, % 左边距
    right=3cm, % 右边距
    top=3cm, % 上边距
    bottom=3cm, % 下边距
}

\title{High-Performance Computing for Cyber-Physical Systems}
\author{Lingjie Zhang}
\date{2024 SS}

\renewcommand{\contentsname}{Contents}

% 修改图片标签名称为 "img"
\renewcommand{\figurename}{Figure}

% 修改表格标签名称为 "table"
\renewcommand{\tablename}{Table}

% 在每个章节重置图片计数器
\counterwithin{figure}{section}

\setlength{\parskip}{1em}
\setlength{\parindent}{0pt} % 取消段落的第一行缩进

\begin{document}
\maketitle

\begin{center}
    \textbf{This course is based on the lecture ED160021 of Technical University of Munich}
\end{center}

This note is a record of some messy knowledge points, which may not be in order, but most of them are parts of my first contact or learning.

\pagenumbering{roman}
\tableofcontents
\newpage
\pagenumbering{arabic}


% Contents from here
% 从这里开始往下写
\section{Best Practice for Resource Cleanup}
\subsection{Context Managers}

Use the \texttt{with} statement to create a context manager that automatically
handles resource cleanup when the block is exited. This is the recommended approach for
file handling, network connections, and other resources that support context managers. For
example:

\begin{python}
    with open("file.txt", "r") as file:
        content = file.read()
    # file is automatically closed when the block is exited
\end{python}

\subsection{Explicit Cleanup Methods}

Some resources may not support context managers, or you may
need more control over when the resource is cleaned up. In such cases, use explicit cleanup
methods provided by the resource, and make sure to call them when the resource is no longer
needed. For example:

\begin{python}
    connection = SomeDatabaseConnection()
    try:
        # Perform database operations
        pass
    finally:
        connection.close()  # Clean up the connection explicitly
\end{python}

\section{Mixin}
A mixin is a special kind of superclass that is designed to provide specific functionality to
other classes through multiple inheritance, without being considered a complete class on its
own. Mixins are usually small and focused, implementing a single piece of functionality that can
be easily combined with other classes. By using mixins, you can create modular and reusable
components that can be easily added to other classes as needed.

A example of using a mixin:
\begin{python}
    class SpeakMixin:
        def speak(self):
            print("Hello!")

    class Person:
        pass
    
    class Employee(Person, SpeakMixin):
        pass

    e = Employee()
    e.speak() # Hello!
\end{python}

\section{Exception classes}
Some common built-in exception classes in Python:

\begin{itemize}
    \item \textbf{SyntaxError}: Indicates a syntax error, typically occurring during the compilation of code when it doesn't conform to Python's syntax rules. \\
    \item \textbf{IndentationError}: Denotes an indentation error, usually occurring when there's incorrect indentation within a code block. \\
    \item \textbf{NameError}: Occurs when trying to reference a variable or function name that hasn't been defined. \\
    \item \textbf{TypeError}: Indicates an error caused by an operation being applied to an object of an inappropriate or incompatible type. \\
    \item \textbf{ValueError}: Indicates an error caused by a value being inappropriate for the operation it's used for. \\
    \item \textbf{AttributeError}: Occurs when attempting to access an attribute that doesn't exist for an object. \\
    \item \textbf{KeyError}: Indicates an error caused by attempting to access a key that doesn't exist in a dictionary. \\
    \item \textbf{IndexError}: Occurs when trying to access an index that doesn't exist in a sequence. \\
    \item \textbf{FileNotFoundError}: Indicates that a file being accessed was not found. \\
    \item \textbf{IOError}: Denotes an I/O error, typically occurring when an input/output operation fails or is interrupted. \\
    \item \textbf{ZeroDivisionError}: Indicates an error caused by attempting to divide by zero. \\
    \item \textbf{OverflowError}: Occurs when a numeric computation result exceeds the representation limit. \\
    \item \textbf{RuntimeError}: Denotes a runtime error, typically occurring when an unrecoverable error condition is encountered during execution. \\
    \item \textbf{StopIteration}: Indicates that an iteration has stopped, often raised when an iterator's \texttt{next()} method has no more elements to return.
\end{itemize}

\section{Encapsulation}
\subsection{Private and Public Attributes}
\begin{itemize}
    \item \textbf{Public attributes}: These attributes are accessible from any part of the code and are intended for
    external use. By default, all attributes in a Python class are public. There is no special naming
    convention for public attributes
    \item \textbf{Private attributes}: These attributes are meant to be used only within the class and are not
    intended for external access. To indicate that an attribute is private, you should prefix its name
    with a double underscore (e.g., \texttt{\_\_private\_attribute}).
\end{itemize}
Here's an example that demonstrates the use of public and private attributes:
\begin{python}
    class BankAccount :
        def __init__ ( self , account_number , initial_balance ) :
            self . account_number = account_number
            self . __balance = initial_balance

        def deposit ( self , amount ) :
            self . __balance += amount

        def withdraw ( self , amount ) :
            if amount <= self . __balance :
                self . __balance -= amount
            else :
                print (" Insufficient funds ")

        def get_balance ( self ) :
            return self . __balance

    account = BankAccount (12345 , 1000)
    print ( account . account_number ) # 12345
    print ( account . get_balance () ) # 1000
    print ( account . __balance ) # AttributeError : 'BankAccount' object has no attribute '__balance '
\end{python}

In this example, the \texttt{BankAccount} class has a public attribute \texttt{account\_number} and a private
attribute \texttt{\_\_balance}. The private attribute can only be accessed and modified within the class,
through methods like \texttt{deposit}, \texttt{withdraw}, and \texttt{get\_balance}. Trying to access the private attribute
directly from outside the class would result in an \texttt{AttributeError}.

\section{Properties and the @property Decorator}
The \texttt{@property} decorator is used to define a method as a property. When a property is accessed
like an attribute, the method decorated with \texttt{@property} is called. This allows us to implement
controlled access to an attribute without the need for explicit getter methods.

To create a setter for a property, we can use the \texttt{@attribute.setter} decorator. This decorator
defines a method as the setter for a property, which is called when the property is assigned a
new value. It provides a clean way to implement validation or other logic when modifying an attribute

Let's rewrite the previous \texttt{BankAccount} example using properties:

\begin{python}
    class BankAccount:
        def __init__(self, balance):
            self._balance = balance

        @property
        def balance(self):
            return self._balance

        @balance.setter
        def balance(self, new_balance):
            if new_balance < 0:
                raise ValueError("Balance cannot be negative.")
            self._balance = new_balance

    account = BankAccount(100)
    print(account.balance) # 100
    account.balance = 200
    print(account.balance) # 200
\end{python}

\section{Abstract Classes and Protocols}
Abstract classes in Python are classes that define a common interface for their subclasses, but
cannot be instantiated themselves.

To define an abstract class in Python, you can use the abc module, which provides the ABC
(Abstract Base Class) metaclass. By inheriting from this metaclass, you can create your own
abstract classes. Here's an example of defining an abstract class:
\begin{python}
    from abc import ABC

    class Shape(ABC):
        pass
\end{python}
In this example, we define an abstract class Shape by inheriting from the ABC metaclass
provided by the abc module. This class can serve as a base class for various shapes, but it cannot
be instantiated directly:
\begin{python}
    shape = Shape() # TypeError : Can ’t instantiate abstract class Shape with abstract methods area
\end{python}

\textbf{Abstract Methods and the @abstractmethod Decorator}

Abstract methods are a way to enforce a common interface for subclasses, ensuring that 
they provide implementations for all required methods.
In Python, you can define abstract methods using the 
@abstractmethod decorator, which is also provided by the \texttt{abc} module.

Here's an example:
\begin{python}
    from abc import ABC, abstractmethod

    class Shape(ABC):

        @abstractmethod
        def area(self):
            pass


    class Circle(Shape):
        def __init__(self, radius):
            self.radius = radius

        def area(self):
            return 3.14159 * self.radius ** 2
\end{python}
If a subclass does not provide an implementation for an abstract method, it will also be 
considered an abstract class, and attempting to instantiate it will result in a \texttt{TypeError}:
\begin{python}
    class Square(Shapre):
        pass

    square = Square() # TypeError: Can't instantiate abstract class Square with abstract methods area
\end{python}

\textbf{Implementing Protocols and Duck Typing}
"Duck Typing" means that you can use an object in a specific context as long as it implements the 
necessary methods or properties, regardless of its actual class. This approach provides great 
flexibility and allows for more generic and reusable code.

To define a protocol in Python 3.8 or later, you can use the \texttt{typing.Protocol} class:
\begin{python}
    from typing import Protocol

    class IterableProtocol(Protocol):
        def iter(self):
    pass
\end{python}
In this example, the \texttt{IterableProtocol} defines a single method, \texttt{iter}, which means 
any class implementing this method would conform to the protocol. The function using this protocol
might look like this:
\begin{python}
    def sum_elements(iterable: IterableProtocol):
        total = 0
        for element in iterable:
            total += element
        return total
\end{python}
This function uses duck typing and the \texttt{IterableProtocol}, so it doesn't care about the 
specific type of the input iterable. It only requires that the input implements the iteration protocol.
This means that the function can work with lists, tuples, sets, and even custom iterable classes:
\begin{python}
    print(sum_elements([1, 2, 3])) # 6
    print(sum_elements((1, 2, 3))) # 6
    print(sum_elements({1, 2, 3})) # 6

    class CustomIterable:
        def iter(self):
            return iter([1, 2, 3])
    
    print(sum_elements(CustomIterable())) # 6
\end{python}

\section{Pythonic Design Patterns}
\subsection{Singleton Pattern}
The Singleton pattern ensures that a class has only one instance and provides a global point of
access to that instance. This pattern is useful when you want to manage a shared resource or
enforce a single point of control over a particular part of your application.

Here's one way to implement the Singleton pattern in Python using a class-level attribute:
\begin{python}
    class Singleton:
        _instance = None

        def __new__(cls, *args, **kwargs):
            if cls._instance is None:
                cls._instance = super().__new__(cls, *args, **kwargs)
            return cls._instance

    singleton1 = Singleton()
    singleton2 = Singleton()

    print(singleton1 is singleton2) # True
\end{python}

\subsection{Factory Pattern}
The Factory pattern is a creational pattern that provides an interface for creating objects in a
superclass but allows subclasses to alter the type of objects that will be created. This pattern is
useful when you want to create objects of different classes, sharing a common interface or base
class, without specifying the exact class to be created.

Here's an example of how to implement the Factory pattern in Python:
\begin{python}
    from abc import ABC, abstractmethod

    class Animal(ABC):
        @abstractmethod
        def speak(self):
            pass

    class Dog(Animal):
        def speak(self):
            return "Woof!"

    class Cat(Animal):
        def speak(self):
            return "Meow!"

    class AnimalFactory:
        @staticmethod
        def create_animal(animal_type):
            if animal_type == "Dog":
                return Dog()
            elif animal_type == "Cat":
                return Cat()
            else:
                raise ValueError("Invalid animal type")

    animal1 = AnimalFactory.create_animal("Dog")
    animal2 = AnimalFactory.create_animal("Cat")

    print(animal1.speak()) # Woof!
    print(animal2.speak()) # Meow!
\end{python}

\subsection{Observer Pattern}
The Observer pattern is a behavioral design pattern that establishes a one-to-many dependency
between objects, where one object (the subject) maintains a list of its dependents (observers) and
notifies them automatically of any changes in state. This pattern is useful when there is a need to
maintain consistency between related objects or to update multiple objects when a single object
changes.

Consider a scenario where we have a data source and multiple components that display or
process the data in different ways. Whenever the data source changes, these components need to
be updated accordingly. The Observer pattern can help us achieve this without tightly coupling
the data source and the components.

Here's an example implementation of the Observer pattern in Python:
\begin{python}
    class Subject:
        def __init__(self):
            self._observers = []

        def attach(self, observer):
            self._observers.append(observer)

        def detach(self, observer):
            self._observers.remove(observer)

        def notify(self):
            for observer in self._observers:
                observer.update(self)

    class DataSource(Subject):
        def __init__(self, value):
            super().__init__()
            self._value = value

        @property
        def value(self):
            return self._value

        @value.setter
        def value(self, new_value):
            self._value = new_value
            self.notify()

    class DisplayObserver:
        def update(self, subject):
            print(f"Display: The new value is {subject.value}")

    class LoggerObserver:
        def update(self, subject):
            print(f"logger: Recorded new value: {subject.value}")

    data_source = DataSource(42)
    display = DisplayObserver()
    logger = LoggerObserver()

    data_source.attach(display)
    data_source.attach(logger)

    data_source.value = 10
\end{python}

\subsection{Decorator Pattern}
Decorators are a powerful and flexible pattern in Python that allows you to modify or enhance
the behavior of functions or classes without altering their code. They are widely used in Python
for a variety of purposes, such as enforcing access control, logging, memoization, and more.

\subsubsection*{Functions Decorating Functions}
Function decorators are a convenient way to wrap a function with additional functionality. They
are typically used when you have a simple use case that doesn’t require maintaining state or
when the decorator logic is straightforward. 

Here's an example of a function decorator:
\begin{python}
    def function_decorator(func):
        def wrapper(*args, **kwargs):
            print("Before calling the original function.")
            result = func(*args, **kwargs)
            print("After calling the original function.")
            return result
        return wrapper

    @function_decorator
    def my_function():
        print("Inside the original function.")

    my_function()
\end{python}

\subsubsection*{Functions Decorating Classes}
Sometimes you may want to apply a decorator to a class instead of a function. This can be useful
when you need to modify or enhance the behavior of an entire class, such as adding or modifying
class methods or attributes. 

Functions can be used to decorate classes, as shown in the following
example:
\begin{python}
    def class_decorator(cls):
        original_method = cls.original_method

        def decorated_method(self):
            print("Before calling the original method.")
            original_method(self)
            print("After calling the original method.")

        cls.decorated_method = decorated_method
        return cls

    @class_decorator
    class MyClass:
        def original_method(self):
            print("Inside the original method.")

    my_instance = MyClass()
    my_instance.decorated_method()
\end{python}

\subsubsection*{Classes Decorating Classes}
When the decorator logic becomes more complex or needs to maintain some state, using a class
decorator can provide greater flexibility and reusability. Class decorators can be extended and
customized through inheritance and can store state as instance variables. 

Here's an example of a class decorator:
\begin{python}
    class ClassDecorator:
        def init(self, cls):
            self._cls = cls

        def __call__(self, *args, **kwargs):
            wrapped_instance = self._cls(*args, **kwargs)
            wrapped_instance.decorated_method = self._decorated_method
            return wrapped_instance

        def _decorated_method(self):
            print("Before calling the original method.")
            self.original_method()
            print("After calling the original method.")

        @ClassDecorator
        class MyClass:
            def original_method(self):
                print("Inside the original method.")

        my_instance = MyClass()
        my_instance.decorated_method()
\end{python}

\subsubsection*{The @ symbol}
It's important to note that the @ symbol used in decorators is merely syntactic sugar that simplifies
the process of applying decorators to functions or classes. Using @decorator before a function
or class definition is equivalent to calling the decorator function with the original function or
class as an argument and then assigning the result back to the original name. 

For example, the following two code snippets are functionally equivalent:
\begin{python}
    @decorator
    def my_function():
        pass

    # is equivalent to

    def my_function():
        pass
    my_function = decorator(my_function)
\end{python}

\subsection{The Strategy Pattern}
The Strategy Pattern is a behavioral design pattern that allows you to define a family of algorithms,
encapsulate each of them in a separate class, and make them interchangeable at runtime. By
using the Strategy Pattern, you can alter the behavior of an object without modifying its code or
structure. This pattern is particularly useful when you need to provide multiple ways to perform
an operation and want to be able to switch between them easily.

Here's an example that demonstrates the Strategy Pattern in Python:
\begin{python}
    from abc import ABC, abstractmethod

    class SortingStrategy(ABC):
        @abstractmethod
        def sort(self, data):
            pass

    class QuickSort(SortingStrategy):
        def sort(self, data):
            # Implementation of the QuickSort algorithm
            pass

    class MergeSort(SortingStrategy):
        def sort(self, data):
            # Implementation of the MergeSort algorithm
            pass

    class Context:
        def init(self, strategy: SortingStrategy):
            self._strategy = strategy

        def set_strategy(self, strategy: SortingStrategy):
            self._strategy = strategy

        def execute_sort(self, data):
            self._strategy.sort(data)

    # Usage example

    data = [5, 3, 1, 4, 2]

    context = Context(QuickSort())
    context.execute_sort(data)

    context.set_strategy(MergeSort())
    context.execute_sort(data)
\end{python}

\end{document}