\documentclass[12pt, a4paper, oneside]{article}
\usepackage{amsmath, amsthm, amssymb, graphicx}
\usepackage{geometry}
\usepackage[bookmarks=true, colorlinks, citecolor=blue, linkcolor=black]{hyperref}
\usepackage{ctex}
\usepackage{tikz}
\usetikzlibrary{patterns}

\geometry{
    a4paper, % 页面尺寸
    left=3cm, % 左边距
    right=3cm, % 右边距
    top=3cm, % 上边距
    bottom=3cm, % 下边距
}

\title{Computer Vision}
\author{Lingjie Zhang}
\date{2024 SS}

\renewcommand{\contentsname}{Contents}

% 修改图片标签名称为 "img"
\renewcommand{\figurename}{Figure}

% 修改表格标签名称为 "table"
\renewcommand{\tablename}{Table}

% 在每个章节重置图片计数器
\counterwithin{figure}{section}

\setlength{\parskip}{1em}

\begin{document}
\maketitle

\centering
\textbf{This course is based on the lecture EI70110 of Technical University of Munich}

\raggedright

\tableofcontents
\pagenumbering{roman}
\newpage
\pagenumbering{arabic}


\section{Wissenswertes über Bilder}

\subsection{Darstellung von Bildern}
\subsubsection*{Von Farbbild zum Intensitätsbild}
\begin{itemize}
    \item Farbbilder bestehen aus mehreren Kanälen
    \item In diesem Kurs ausschließlich Graustufenbilder
\end{itemize}

\begin{figure}[htbp]
    \centering
    \includegraphics[scale=0.8]{../img/1-1.png}
    \caption{RBG image}
    \label{img/1-1}
\end{figure}

\subsubsection*{Kontinuierliche und diskrete Darstellung}
\begin{itemize}
    \item \textbf{Kontinuierliche} Darstellung als Funktion zweier Veränderlicher (zum Herleiten von Algorithmen) 
    $I: \mathbb{R}^2\supset\Omega\to\mathbb{R}, (x,y)\mapsto I(x,y)$
    \item Häufige Annahmen:
    \begin{itemize}
        \item $I$ differenzierbar
        \item $\Omega$ einfach zusammenhängend und beschränkt
    \end{itemize}
    \item Diskrete Darstellung als Matrix $I\in \mathbb R^{m\times n}$
    
    Eintrag $I_{k,l}$ entspricht dem Intensitätswert
    \item Skalierung typischerweise zwischen [0, 255] oder [0, 1]
\end{itemize}

\begin{table}[h]
\centering
\begin{tabular}{|c|}
\hline
VGA: 480$\times$ 640  Pixel (ca. 0.3 Megapixel) \\
\hline
HD: 720$\times$ 1280  Pixel (ca. 1.0 Megapixel) \\
\hline
FHD:  1080$\times$ 1920  Pixel (ca. 2.1 Megapixel) \\
\hline
\end{tabular}
\label{tab:table_label}
\end{table}

\begin{figure}[htbp]
    \centering
    \includegraphics[scale=0.7]{../img/1-2.png}
    \caption{Graph einer Funktion}
    \label{img/1-2}
\end{figure}

\begin{figure}[htbp]
    \centering
    \includegraphics[scale=0.7]{../img/1-3.png}
    \caption{Graph eines Fotos}
    \label{img/1-3}
\end{figure}

\subsubsection*{Diskretes Abtasten}
\begin{itemize}
    \item Abtasten eines eindimensionalen Signals
    $$S\{f(x)\}=(...,f(x-1),f(x),f(x+1),...)$$
    \item Abtasten eines Bildes \\
    $$
    S\{I(x,y)\}=
    \begin{bmatrix}
    \ddots & \vdots & \vdots & \vdots & \\\
    \cdots & I(x-1,y-1) & I(x-1,y) & I(x-1,y+1) & \cdots\\
    \cdots & I(x,y-1) & I(x,y) & I(x,y+1) & \cdots\\
    \cdots & I(x+1,y-1) & I(x+1,y) & I(x+1,y+1) & \cdots\\
    & \vdots & \vdots & \vdots & \ddots
    \end{bmatrix}
    $$
\end{itemize}

\subsubsection*{Diskrete Darstellung/Matrixdarstellung}
\begin{itemize}
    \item Annahme: Ursprung links oben
    \item Matrixeintrag ist $I_{k,l}=S\{I(0,0)\}_{kl}$
\end{itemize}
\begin{figure}[htbp]
    \centering
    \includegraphics{../img/1-4.png}
    \caption{Diskrete Darstellung/Matrixdarstellung}
    \label{img/1-4}
\end{figure}

\subsubsection*{Zusammenfassung}
\begin{itemize}
    \item Bilder in Grautönen
    \item Bilder als Matrizen
    \item Bilder als glatte Funktionen
\end{itemize}

\newpage

\subsection{Bildgradient}
\subsubsection{Der Gradient eines Bildes}
\begin{figure}[htbp]
    \centering
    \includegraphics{../img/1-5.png}
    \caption{Der Gradient eines Bildes}
    \label{img/1-5}
\end{figure}
Kanten sind starke lokale Änderungen der Intensität. Lokale Änderungen werden durch den Gradienten beschrieben.
$$
    \nabla I(x,y)=\begin{bmatrix}
    \frac{\partial}{\partial x}I(x,y) \\
    \frac{\partial}{\partial y}I(x,y)
    \end{bmatrix}
$$

\paragraph{Wie schätzt man den Gradienten?}
Gegeben ist das Bild in diskreter Form $I\in \mathbb R^{m\times n}$.
$$
    \frac{\partial}{\partial x} I(x,y)\approx I(x+1,y)-I(x,y)
$$
$$
    \frac{\partial}{\partial y}I(x,y)\approx I(x,y+1)-I(x,y)
$$

\subsubsection{Diskretes und kontinuierliches Signal}
\paragraph*{Interpolation}
Vom diskreten Signal $f[x]=S\{f(x)\}$ zum kontinuierlichen Signal $f(x)$. Interpoliertes Signal ist Faltung der Abtastwerte mit dem Interpolationsfilter.
$$
    f(x)\approx \sum_{k=-\infty}^{\infty}f[k]h(x-k)=:f[x]*h(x)
$$

\paragraph*{Interpolationsfilter}
Diskretes Signal: $f[x]=S\{f(x)\}$. Kontinuierliches Signal: $f(x)\approx f[x]*h(x)$.
\begin{itemize}
    \item Gaußfilter: $h(x)=g(x)$
    \item Ideales Interpolationsfilter: $h(x)=\operatorname{sinc}(x)$
    \item Damit gilt: $f[x]*h(x)=f(x)$
\end{itemize}

\begin{figure}[htbp]
    \centering
    \includegraphics[scale=1]{../img/1-6.png}
    \caption{$g(x)=Ce^{\frac{-x^2}{2\sigma^2}}$}
    \label{img/1-6}
\end{figure}
\begin{figure}[htbp]
    \centering
    \includegraphics[scale=1]{../img/1-7.png}
    \caption{$sinc(x)=\frac{sin(\pi x)}{\pi x}, sinc(0):=1$}
    \label{img/1-7}
\end{figure}

\subsubsection{Die diskrete Ableitung}
\paragraph*{Mit Hilfe des rekonstruierten Signals}
\begin{itemize}
    \item Algorithmisch
    \begin{enumerate}
        \item Rekonstruktion des kontinuierlichen Signals
        \item Ableitung des kontinuierlichen Signals
        \item Abtastung der Ableitung
    \end{enumerate}
    \item Herleitung
    \begin{itemize}
        \item $f'(x)\approx \frac{d}{dx}(f[x]*h(x))$\\$f[x]*h'(x)$
        \item $f'[x]=f[x]*h'[x]$\\$=\sum\limits _{k=-\infty}^{\infty}f[x-k]h'[k]$
    \end{itemize}
\end{itemize}

\begin{figure}[htbp]
    \centering
    \includegraphics[scale=1]{../img/1-8.png}
    \caption{Sinc-Funktion(Langsames Abklingen)}
    \label{img/1-8}
\end{figure}
\begin{figure}[htbp]
    \centering
    \includegraphics[scale=1]{../img/1-9.png}
    \caption{Gaußfilter(Schnelles Abklingen)}
    \label{img/1-9}
\end{figure}

\subsubsection{Zweidimensionale Rekonstruktion}
\paragraph*{Separables 2D-Gaußfilter}
2D-Rekonstruktion: $I(x,y)\approx I[x,y]*h(x,y)=\sum\limits _{k=-\infty}^{\infty}\sum\limits_{l=-\infty}^{\infty}I[k,l]g(x-k)g(y-l)$
\begin{figure}[htbp]
    \centering
    \includegraphics[scale=1]{../img/1-10.png}
    \caption{$h(x,y):=g(x)g(y)$}
    \label{img/1-10}
\end{figure}

\subsubsection{Zweidimensionale Ableitung}
\paragraph*{Ausnutzen der Separabilität}
Ableitung in x-Richtung
\begin{align*}
    \frac{d}{dx}I(x,y) & \approx I[x,y]*(\frac{d}{dx}h(x,y)) \\
    & = \sum\limits_{k=-\infty}^{\infty} \sum\limits_{l=-\infty}^{\infty}I[k,l]g'[x-k]g(y-l)
\end{align*}
\begin{align*}
    S\{\frac{d}{dx}I(x,y)\} & = I[x,y]*g'[x]*g[y] \\
    & = \sum\limits_{k=-\infty}^{\infty} \sum\limits_{l=-\infty}^{\infty}I[x-k,y-l]g'[k]g[l]
\end{align*}

\subsubsection{Endliche Approximation des Gaußfilters}
\paragraph*{Normierung des endlichen Filters}
\begin{itemize}
    \item In der Praxis wird die unendliche Summe durch wenige Summanden approximiert
    \item Wie wählt man eine geeignete Gewichtung $C$ des Gaußfilters $g(x)=Ce^{\frac{-x^2}{2\sigma^2}}$?
    \item Interpoliertes Signal: $f(x)\approx \sum\limits_{k=-\infty}^{\infty}f[k]g(x-k)$
    \item Abgetastetes interpoliertes Signal: $f[x]\approx \sum\limits_{k=-\infty}^{\infty}f[x-k]g[k]$
    \item Approximation durch endliche Summe: $f[x]\approx \sum\limits_{k=-n}^{n}f[x-k]g[k]$
    \item Die endliche Approximation von $f[x]$ ist eine gewichtete Summe der Werte $f[x-n],...,f[x+n]$ mit den Gewichten $g[n],...,g[-n]$
    \item Normierungskonstante $C$ so gewählt, dass sich alle Gewichte zu 1 addieren
    \item Wähle $C=\frac{1}{\sum\limits_{-n\le k\le n}e^{\frac{-k^2}{2\sigma^2}}}$
\end{itemize}

\subsubsection{Sobel-Filter}
\paragraph*{Herleitung}
\begin{itemize}
    \item Approximation von $S\{\frac{d}{dx}I(x,y)\}=I[x,y]*g'[x]*g[y]=\sum\limits_{k=-\infty}^{\infty}\sum\limits_{l=-\infty}^{\infty}I[x-k,y-l]g'[k]g[l]$ durch endliche Summe $\sum\limits_{k=-1,0,1}^{}\sum\limits_{l=-1,0,1}^{}I[x-k,y-l]g'[k]g[l]$
    \item Daraus folgt der Normierungsfaktor $C=\frac{1}{1+2e^{-\frac{1}{2\sigma^2}}}$
    \item Für die Wahl $\sigma = \sqrt[]{\frac{1}{2\log 2}}$ ergeben sich somit die Werte
    \begin{align*}
        & g[-1]=\frac{1}{4}; g[0]=\frac{1}{2}; g[1]=\frac{1}{4} \\
        & g'[-1]=\frac{1}{2}\log 2; g'[0]=0; g'[1]=-\frac{1}{2}\log 2\;\;\;\;(\frac{1}{2}\log 2\approx 0.35)
    \end{align*}
    \item Aus praktischen Gründen sind ganzzahlige Filterkoeffizienten erwünscht
    \item Für das Detektieren von Intensitätsunterschieden ist ein Vielfaches des Gradienten ausreichend
\end{itemize}

\begin{figure}[htbp]
    \centering
    \includegraphics[scale=1]{../img/1-11.png}
    \label{img/1-11}
\end{figure}

\paragraph*{Beispiel}
\begin{figure}[!h]
    \centering
    \includegraphics[scale=0.6]{../img/1-12.png}
    \caption{Sobel-Filterung}
    \label{img/1-12}
\end{figure}

\subsubsection{Zusammenfassung}
\begin{itemize}
    \item Der Bildgradient ist ein wichtiges Werkzeug für die Bestimmung von lokalen Intensitätsänderungen
    \item Diskrete Ableitung wird durch Differenzieren des interpolierten Signals berechnet
    \item Sobel-Filter sind ganzzahlige Approximation eines Vielfachen des Gradienten 
\end{itemize}

\subsection{Merkmalspunkte-Ecken und Kanten}
\subsubsection{Ecken und Kanten}
\paragraph*{...liefern markante Bildmerkmale}
\begin{itemize}
    \item Bestimmung von Konturen
    \item Berechnungen von Bewegungen in Bildsequenzen
    \item Schätzen von Kamerabewegung
    \item Registrierung von Bildern
    \item 3D-Rekonstruktion
\end{itemize}

\subsubsection{Harris Ecken- und Kantendetektor}
\paragraph*{Änderung des Bildsegments in Abhängigkeit der Verschiebung}
\begin{figure}[htbp]
    \centering
    \includegraphics[scale=1]{../img/1-13.png}
    \label{img/1-13}
\end{figure}
\begin{itemize}
    \item Ecke: Verschiebung in jede Richtung bewirkt Änderung
    \item Kante: Verschiebung in jede bis auf genau eine Richtung bewirkt Änderung
    \item Homogene Fläche: Keine Änderung, egal in welche Richtung
\end{itemize}
\paragraph*{Formelle Beschreibung der Änderung}
\begin{itemize}
    \item Position im Bild: 
    $x=\begin{bmatrix}
        x_1 \\
        x_2
    \end{bmatrix},\;\;\;\;
    I(x)=I(x_1,x_2)$
    \item Verschiebungsrichtung: $u=\begin{bmatrix}
        u_1 \\
        u_2
    \end{bmatrix}$
    \item Änderung des Bildsegments:
    $$
    S(u)=\int_W (I(x+u)-I(x))^2dx
    $$
    \item Differenzierbarkeit von $I$:
    $$
    \lim_{u\to 0}\frac{I(x+u)-I(x)-\nabla I(x)^\top u}{||u||}=0
    $$
\end{itemize}
\paragraph*{Approximation der Änderung}
\begin{itemize}
    \item Folgerung aus Differenzierbarkeit: $I(x+u)-I(x)=\nabla I(x)^\top u + o(||u||)$
    \item Restterm $o(||u||)$ mit der Eigenschaft $\lim_{u\to 0}o(||u||)/||u||=0$
    \item Approximation für kleine Verschiebungen: $I(x+u)-I(x)\approx \nabla I(x)^\top u$
    \item Approximation der Änderung im Bildsegment:
    $$
    S(u)=\int_W(I(x+u)-I(x))^2dx\approx\int_W(\nabla I(x)^\top u)^2dx
    $$
\end{itemize}
\paragraph*{Die Harris-Matrix}
\begin{itemize}
    \item Ausmultiplizieren des Integrals:
    $$
    S(u)=\int_W(\nabla I(x)^\top u)^2dx=u^\top (\int_W \nabla I(x)\nabla I(x)^\top dx)u
    $$
    \item Harris-Matrix: $G(x)=\int_W \nabla I(x)\nabla I(x)^\top ds$
    $$\nabla I(x)\nabla I(x)^\top =\begin{bmatrix}
        (\frac{\partial}{\partial x_1}I(x))^2 & \frac{\partial}{\partial x_1}I(x)\frac{\partial}{\partial x_2}I(x) \\
        \frac{\partial}{\partial x_2}I(x)\frac{\partial}{\partial x_1}I(x) & (\frac{\partial}{\partial x_2}I(x))^2
    \end{bmatrix}
    $$
    \item Approximative Änderung des Bildsegments:
    $$
    S(u)\approx u^\top G(x)u
    $$
\end{itemize}

\paragraph*{Eigenwertzerlegung}
\begin{itemize}
    \item Eigenwertzerlegung der Harris-Matrix:
    $$
    G(x)=\int_W\nabla I(x)\nabla I(x)^\top dx=V \begin{bmatrix}
        \lambda_1 & \\
         & \lambda_2
    \end{bmatrix}
    V^\top
    $$
    mit $VV^\top =I_2$ und den Eigenwerten $\lambda_1\ge \lambda_2\ge 0$
    \item Änderung in Abhängigkeit der Eigenvektoren: $V=[v_1,v_2]$
    $$
    S(u)\approx u^\top G(x)u=\lambda_1(u^\top v_1)^2+\lambda_2(u^\top v_2)^2
    $$
\end{itemize}

\paragraph*{Art des Merkmals in Abhängigkeit der Eigenwerte}
\begin{itemize}
    \item Beide Eigenwerte positive
    \begin{itemize}
        \item $S(u)>0$ für alle $u$ (Änderung in jede Richtung)
        \item Untersuchtes Bildsegment enthält eine Ecke
    \end{itemize}
    \item Ein Eigenwert positiv, ein Eigenwert gleich null
    \begin{itemize}
        \item $S(u)\left\{\begin{matrix}
           =0,&\text{falls}&; u=rv_2\;\;&\text{(Keine Änderung nur in Richtung}\\ 
           &&&\text{des Eigenvektors zum Eigenwert 0)}  \\
           >0,&\text{sonst}&
           \end{matrix}\right.$
        \item Untersuchtes Bildsegment enthält eine Kante
    \end{itemize}
    \item Beide Eigenwerte gleich null
    \begin{itemize}
        \item $S(u)=0$ für alle $u$ (Keine Änderung, egal in welche Richtung)
        \item Untersuchtes Bildsegment ist eine homogene Fläche
    \end{itemize}
\end{itemize}

\subsubsection{Praktische Realisierung des Harris-Detektors}
\paragraph*{Berechnung der Harris-Matrix}
\begin{itemize}
    \item Approximiere $G(x)$ durch endliche Summe
    $$
    G(x)=\int_W\nabla I(x)\nabla I(x)^\top dx\approx \sum\limits_{\tilde{x}\in W(x)}\nabla I(\tilde{x})\nabla I(\tilde{x})^\top
    $$
    \item Gewichtete Summe in Abhängigkeit der Position von $\tilde{x}$
    $$
    G(x)\approx \sum\limits_{\tilde{x}\in W(x)}w(\tilde{x})\nabla I(\tilde{x})\nabla I(\tilde{x})^\top
    $$
    \item Gewichte $w(\tilde{x})>0$ betonen Einfluss der zentralen Pixel
\end{itemize}
\paragraph*{Eigenwerte}
\begin{itemize}
    \item In der Realität nehmen Eigenwerte nie genau den Wert Null an, z.B. auf Grund von Rauschen, diskreter Abtastung und numerischen Ungenauigkeiten
    \item Charakteristik in der Praxis
    \begin{itemize}
        \item Ecke: zwei große Eigenwerte
        \item Kante: ein großer Eigenwert, ein kleiner Eigenwert
        \item Homogene Fläche: zwei kleine Eigenwerte
    \end{itemize}
    \item Entscheidung mittels empirischer Schwellwerte
\end{itemize}
\paragraph*{Ein einfaches Kriterium für Ecken und Kanten}
\begin{itemize}
    \item Betrachte die Größe $H:=\det(G)-k(\text{tr}(G))^2=(1-2k)\lambda_1\lambda_2-k(\lambda_1^2+\lambda_2^2)$
    \item Ecke (beide Eigenwerte groß)
    \begin{itemize}
        \item $H$ größer als ein positiver Schwellwert
    \end{itemize}
    \item Kante (ein Eigenwert groß, ein Eigenwert klein)
    \begin{itemize}
        \item $H$ kleiner als ein negativer Schwellwert
    \end{itemize}
    \item Homogene Fläche (beide Eigenwerte klein)
    \begin{itemize}
        \item $H$ betragsmäßig klein
    \end{itemize}
\end{itemize}

\subsubsection{Zusammenfassung}
\paragraph*{Harris-Detektor zur Bestimmung von Merkmalspunkten}
\begin{itemize}
    \item Auswertung der (approximierten) Harris-Matrix
    $$
    G(x)\approx \sum\limits_{\tilde{x}\in W(x)}w(\tilde{x})\nabla I(\tilde{x})\nabla I(\tilde{x})^\top
    $$
    \item Eigenwertzerlegung von $G(x)$ liefert auch Info über Richtung etwaiger Kanten 
    \item Effiziente Implementierung mit Hilfe des Ausdrucks
    $$
    H:=\det(G)-k(\text{tr}(G))^2
    $$
    \item Entscheidung mittels Schwellwerten
    \begin{itemize}
        \item Ecke: $0<\tau_+ <H$
        \item Kante: $H<\tau_- <0$
        \item Homogene Fläche: $\tau_- <H< \tau_+$
    \end{itemize}
\end{itemize}

\subsection{Korrespondenzschätzung für Merkmalspunkte}

\subsubsection{Korrespondenzschätzung}
\paragraph*{Problemstellung}
\begin{itemize}
  \item Gegeben sind zwei Bilder $I_1:\Omega_1\to\mathbb{R}, I_2:\Omega_2\to\mathbb{R}$ derselben 3D-Szene
  \item Finde Paare von Bildpunkten $(x^{(i)},y^{(i)})\in \Omega_1\times\Omega_2$, die zu gleichen 3D-Punkten korrespondieren
  \begin{figure}[htbp]
      \centering
      \includegraphics[scale=0.6]{../img/1-20.png}
      \label{img/1-20}
  \end{figure}
\end{itemize}

\begin{itemize}
  \item In dieser Session: Korrespondenzen für Merkmalspunkte in $I_1$ und $I_2$
  \item Habe Merkmalspunkte $\{x_1,...,x_n\}\subset\Omega_1$ und $\{y_1,...,y_n\}\subset\Omega_2$
  \item Finde passende Paare von Merkmalspunkten
\end{itemize}

\subsubsection{Sum of squared differences (SSD)}
\paragraph*{Formelle Beschreibung}
\begin{itemize}
  \item Betrachte Bildausschnitte $V_i$ um $x_i$ und $W_i$ um $y_i$ in Matrixdarstellung und vergleiche die Intensitäten
  \begin{figure}[htbp]
      \centering
      \includegraphics[scale=0.6]{../img/1-14.png}
      \label{img/1-14}
  \end{figure}
  \item Ein Kriterium: $d(V,W)=||V-W||_F^2$
  \item Dabei ist $||A||_F^2=\sum\limits_{kl}A_{kl}^2$ die quadrierte Frobeniusnorm
  \item Finde zu $V_i$ das $W_j$ mit $j=\arg_{k=1,...,n}\min d(V_i,W_k)$
  \item Annahme: Wenn $W_j$ zu $V_i$ passt, dann auch umgekehrt
\end{itemize}
\paragraph*{Änderungen der Beleuchtung oder Drehungen}
\begin{figure}[htbp]
    \centering
    \includegraphics[scale=0.6]{../img/1-15.png}
    \label{img/1-15}
\end{figure}
\begin{itemize}
  \item Normierung von Intensität und Orientierung benötigt!
\end{itemize}

\subsubsection{Rotationsnormierung}
\paragraph*{mittels Gradientenrichtung}

\begin{itemize}
  \item Vorverarbeitung:
    \begin{enumerate}
      \item Bestimme Gradienten in allen Merkmalspunkten.
      \item Rotiere Regionen um Merkmalspunkte so, dass Gradient in eine Richtung zeigt.
      \item Extrahiere $V,W$ aus rotierten Regionen.
    \end{enumerate}
\end{itemize}
\begin{figure}[htbp]
    \centering
    \includegraphics[scale=0.6]{../img/1-16.png}
    \label{img/1-16}
\end{figure}
\begin{figure}[htbp]
    \centering
    \includegraphics[scale=0.6]{../img/1-17.png}
    \label{img/1-17}
\end{figure}
\begin{figure}[htbp]
    \centering
    \includegraphics[scale=1]{../img/1-18.png}
    \label{img/1-18}
\end{figure}

\subsubsection{Bias and Gain Modell}
\paragraph*{Modellierung von Kontrast und Helligkeit}

\begin{itemize}
  \item Skalierung der Intensitätswerte (Gain) mit $\alpha$
  \item Verschiebung der Intensitätswerte (Bias) mit $\beta$.
  \item Gain-Modell: $W\approx\alpha V$
  \item Bias-Modell: $W\approx V+\beta\mathbf{1}\mathbf{1}^\top\;\;\;\;\mathbf{1}=(1,...,1)^\top$ 
  \item Bias-and-Gain Modell: $W\approx\alpha V+\beta \mathbf{1} \mathbf{1}^\top$
\end{itemize}
\begin{figure}[htbp]
    \centering
    \includegraphics[scale=0.6]{../img/1-19.png}
    \label{img/1-19}
\end{figure}

\paragraph*{Berechnung des Mittelwerts}
\begin{itemize}
  \item Mittelwertbildung der Intensität
  \begin{align*}
      \bar W & = \frac{1}{N}(\mathbf{1}\mathbf{1}^\top W\mathbf{1}\mathbf{1}^\top) \\
      & \approx \frac{1}{N}(\mathbf{1}\mathbf{1}^\top (\alpha V+\beta \mathbf{1}\mathbf{1}^\top)\mathbf{1}\mathbf{1}^\top) \\
      & = \alpha\frac{1}{N}(\mathbf{1}\mathbf{1}^\top V\mathbf{1}\mathbf{1}^\top)+\beta\mathbf{1}\mathbf{1}^\top \\
      & = \alpha\bar V+\beta\mathbf{1}\mathbf{1}^\top
  \end{align*}
  \item Subtraktion der Mittelwertmatrix
  \begin{align*}
      W-\bar W & \approx\alpha V+\beta\mathbf{1}\mathbf{1}^\top-(\alpha\bar V+\beta\mathbf{1}\mathbf{1}^\top) \\
      & =\alpha(V-\bar V)
  \end{align*}
\end{itemize}

\paragraph*{Berechnung der Standardabweichung}
\begin{itemize}
  \item Standardabweichung der Intensität
  \begin{align*}
      \sigma(W) & =\;\sqrt[]{\frac{1}{N-1}||W-\bar W||_F^2} \\
      & =\;\sqrt[]{\frac{1}{N-1}\text{tr}((W-\bar W)^]\top (W-\bar W))} \\
      & \approx \;\sqrt[]{\frac{1}{N-1}\text{tr}(\alpha(V-\bar V)^\top\alpha(V-\bar V))} \\
      & =\;\alpha\sigma(V)
  \end{align*}
\end{itemize}

\paragraph*{Kompensation von Bias und Gain}
\begin{itemize}
    \item Normalisierung der Bildsegmente durch
    \begin{itemize}
        \item Subtraktion des Mittelwertes.
        \item Division durch Standardabweichung.
        \begin{align*}
            W_n:= & \;\frac{1}{\sigma(W)}(W-\bar W) \\
            \approx & \;\frac{1}{\alpha\sigma(V)}(\alpha(V-\bar V)) \\
            = & \;\frac{1}{\sigma (V)}(V-\bar V) \\
            = & \;V_n
        \end{align*}
    \end{itemize}
\end{itemize}

\subsubsection{Normalized Cross Correlation (NCC)}
\paragraph*{Herleitung aus SSD}
\begin{itemize}
  \item SSD von zwei normalisierten Bildsegmenten $||V_n-W_n||_F^2=2(N-1)-2\text{tr}(W_n^\top V_n)$
  \item Die Normalized Cross Correlation der beiden Bildsegmente ist definiert als $\frac{1}{N-1}\text{tr}(W_n^\top V_n)$
  \item Es gilt -1$\le$NCC$\le$1
  \item Zwei normalisierte Bildsegmente sind sich ähnlich, wenn
  \begin{itemize}
      \item SSD klein (wenig Unterschiede)
      \item NCC nahe bei +1 (hohe Korrelation)
  \end{itemize}
\end{itemize}

\subsubsection{Zusammenfassung}

\begin{itemize}
  \item Finde Merkmale in Bild 1 und Bild 2
  \item Kompensiere Rotation durch Ausrichten des Gradienten für jeden Merkmalspunkt
  \item Extrahiere Bildsegment um jeden Merkmalspunkt.
  \item Beleuchtungskompensation durch Normierung der Bildsegmente.
  \item Vergleiche die normalisierten Bildsegmente durch SSD oder NCC.
\end{itemize}

\newpage

\section{Bildentstehung}

\subsection{Das Lochkameramodell}

\subsubsection{Abbildung durch eine dünne Linse}

\begin{itemize}
    \item Strahlen parallel zur optischen Achse werden so gebrochen, dass sie durch den Brennpunkt gehen.
    \item Strahlen durch das optische Zentrum werden nicht abgelenkt.
\end{itemize}

\begin{figure}[!h]
    \centering
    \includegraphics[width=0.5\textwidth]{../img/2-1.png}
    \label{img/2-1}
\end{figure}

\begin{figure}[!h]
    \centering
    \includegraphics[width=0.5\textwidth]{../img/2-2.png}
    \label{img/2-2}
\end{figure}

\paragraph*{Herleitung der Abbildungsgleichung}

\begin{itemize}
    \item $|\frac{b}{B}| = \frac{z-f}{f}$
    \item $|\frac{z}{Z}| = |\frac{b}{B}|$
    \item Gleichung für dünne Linsen: $\frac{1}{|Z|} + \frac{1}{z} = \frac{1}{f}$
\end{itemize}

\subsubsection{Abbildung durch eine Lochkamera}

\paragraph*{Idealisierte Annahmen}

\begin{itemize}
    \item Sehr kleine Öffnung vor der Linse
    \item Beliebig großer Blickwinkel
    \item Bild wird scharf auf Brennebene abgebildet
    \item Es gilt: $b = -\frac{fB}{Z}$
\end{itemize}

\paragraph*{Abbildung eines Punktes im Raum auf die Brennebene}

\begin{figure}[!h]
    \centering
    \includegraphics[width=0.5\textwidth]{../img/2-3.png}
    \label{img/2-3}
\end{figure}

\begin{itemize}
    \item Koordinaten von $P$ bzgl. optischem Zentrum der Kamera seien $X,Y,Z$
    \item Dann sind die Koordinaten des Bildpunktes $(-\frac{fX}{Z}, -\frac{fY}{Z}, -f)$
\end{itemize}

\subsubsection{Frontales Lochkameramodell}

\paragraph*{Projektion eines Punktes}

\begin{figure}[!h]
    \centering
    \includegraphics[width=0.5\textwidth]{../img/2-4.png}
    \label{img/2-4}
\end{figure}

\begin{itemize}
    \item Koordinaten des Bildpunktes beim frontalen Lochkameramodell $(\frac{fX}{Z}, \frac{fY}{Z}, f)$
    \item Ideale perspektivische Projektion: 
    $$
    \pi: \mathbb{R}^3\backslash \{(X,Y)-\text{Ebene}\}\to \mathbb{R}^2, \begin{bmatrix}
        X \\
        Y \\
        Z
    \end{bmatrix} 
    \mapsto \frac{f}{Z}\begin{bmatrix}
        X \\
        Y
    \end{bmatrix}
    $$
\end{itemize}

\subsubsection{Abbildung durch eine Lochkammera}

\paragraph*{Idealisierte Annahmen}

\begin{figure}[!h]
    \centering
    \includegraphics[width=0.5\textwidth]{../img/2-5.png}
    \label{img/2-5}
\end{figure}

\begin{itemize}
    \item Dünne Linse
    \item Kleine Blende
    \item Beliebig großer Blickwinkel
    \item Perspektivische Projektion:
    $$
    \pi: \mathbb{R}^3\backslash \{(X,Y)-\text{Ebene}\}\to \mathbb{R}^2, \begin{bmatrix}
        X \\
        Y \\
        Z
    \end{bmatrix} 
    \mapsto \frac{f}{Z}\begin{bmatrix}
        X \\
        Y
    \end{bmatrix}
    $$
\end{itemize}

\subsubsection{Zusammenfassung}

\begin{itemize}
    \item Lochkameramodell
    \item Ideale perspektivische Projektion: die Abbildung der Koordinaten des 3D Punktes 
    auf die 2D-Koordinaten in der Brennebene
\end{itemize}

\subsection{Homogene Koordinaten}

\subsubsection{Wiederholung: Lochkameramodell}

\paragraph*{Bildpunkte und Geraden}

\begin{itemize}
    \item Alle Punkte auf einer Geraden durch das optische Zentrum werden auf denselben Bildpunkt abgebildet. 
    \item Umgekehrt existiert zu jedem Bildpunkt genau eine Gerade. Dies wird im Lochkameramodell wiederholt.
\end{itemize}

\subsubsection{Der projektive Raum}

\begin{itemize}
    \item Zwei Vektoren $x,y\in\mathbb{R}^n$ nennen wir zueinander \textbf{äquivalen}t, 
    falls ein $\lambda\neq 0$ existiert mit \( \lambda \neq 0 \). 
    In diesem Fall schreiben wir $$x\sim y$$
    \item Eine Gerade durch $x\neq 0$ können wir nun beschreiben als \textbf{Äquivalenzklasse}
    $$[x]:=\{y\in\mathbb{R}^n|y\sim x\}$$
    \item Die Menge aller Geraden im Raum heißt projektiver Raum.
    $$\mathbb{P}_n = \{[x]|x\in\mathbb{R}^{n+1}\backslash \{0\}\}$$
\end{itemize}

\subsubsection{Homogene Koordinaten}

\begin{figure}[!h]
    \centering
    \includegraphics[width=0.5\textwidth]{../img/2-6.png}
    \label{img/2-6}
\end{figure}

\begin{itemize}
    \item Normiert man die Längeneinheit auf die Brennweite, ist die Bildebene gegeben durch
    $$BE=
    \begin{Bmatrix}
        \begin{bmatrix}
            X \\
            Y \\
            Z
        \end{bmatrix}\in\mathbb{R}^3\;|\;Z=1 \\
    \end{Bmatrix}\
    $$
    \item Der Vektor $\begin{bmatrix}
        X \\
        Y \\
        1
    \end{bmatrix}$ heißt die \textbf{homogenen Koordinaten} von $\begin{bmatrix}
        X \\
        Y
    \end{bmatrix}$
    \item Allgemeine Definition: $x:=(X_1,\dots,X_n)^T\in\mathbb{R}^n$.
    Dann heißt $x^{(hom)}:=(X_1,\dots,X_n,1)^T\in\mathbb{R}^{n+1}$ die homogenen Koordinaten von $x$
\end{itemize}

\subsubsection{Zusammenfassung}

\begin{itemize}
    \item Äquivalente Vektoren unterscheiden sich nur durch Multiplikation mit einer Zahl ungleich Null.
    \item Homogene Koordinaten eines Vektors erhält man durch Hinzufügen einer weiteren Koordinate mit dem Wert Eins.
\end{itemize}

\subsection{Euklidische Bewegungen}

\subsubsection{Motivation}

\begin{itemize}
    \item Zwei Bilder einer 3D-Szene aus verschiedenen Positionen.
    \item Positionswechsel bestehen aus Rotation der Kamera und anschließender Translation.
    \item Beschreibung der Kamerabewegung in Form von Koordinatenänderung eines festen Raumpunktes.
\end{itemize}

\begin{figure}[!h]
    \centering
    \includegraphics[width=0.5\textwidth]{../img/2-7.png}
    \label{img/2-7}
\end{figure}

\subsubsection{Visualisierung}

\textbf{Translation}

\begin{figure}[!h]
    \centering
    \includegraphics[width=0.36\textwidth]{../img/2-8.png}
    \label{img/2-8}
\end{figure}

\vspace{1em}
\textbf{Rotation}

\begin{figure}[!h]
    \centering
    \includegraphics[width=0.36\textwidth]{../img/2-9.png}
    \label{img/2-9}
\end{figure}


\space
\subsubsection{Euklidische Bewegungen}

\paragraph*{Allgemein}

\begin{itemize}
    \item Die Matrizen $O(n):=\{O\in\mathbb{R}^{n\times n}|O^TO=I_n\}$ heißen orthogonale Matrizen.
    \item Rotationen im $\mathbb{R}^n$ Raum werden beschrieben durch die speziellen orthogonalen Matrizen.
    $$SO(n):=\{R\in\mathbb{R}^{n\times n}|R^TR=I_n, \det (R)=1\}$$
    \item Euklidische Bewegungen durch die Koordinatenänderung $$g_{R,T}:
    \mathbb{R}^n\to\mathbb{R}^n, \mathbf{P}\mapsto R\mathbf{P}+T$$
    \item Seien $\mathbf{P}_1$ die Koordinaten eines Punktes bzgl. 
    CF 1 und $\mathbf{P}_2$ die Koordinaten des selben Punktes bzgl. 
    des bewegten CF 2, dann ist $\mathbf{P}_2=R\mathbf{P}_1+T$
\end{itemize}

\paragraph*{In homogenen Koordinaten}
\begin{itemize}
    \item In homogenen Koordinaten als Matrix-Vektor-Multiplikation
    $$M=\begin{bmatrix}
        R & T \\
        0 & 1
    \end{bmatrix}\in\mathbb{R}^{(n+1)\times(n+1)}, R\in SO(n), T\in \mathbb{R}^n$$
    $$\mathbf{P}_2^{(hom)}=M\mathbf{P}_1^{hom}$$
    \item Spezielle Euklidische Gruppe
    $$SE(n) = \{M=\begin{bmatrix}
        R & T \\
        0 & 1
    \end{bmatrix}|R\in SO(n), T\in\mathbb{R}^n\}\subset\mathbb{R}^{(n+1)\times (n+1)}$$
\end{itemize}

\paragraph*{Eigenschaften}

\begin{itemize}
    \item Verknüpfungen zweier eukl. Bewegungen ist wieder eine eukl. Bewegung 
    $$M_1M_2 = \begin{bmatrix}
        R_1 & T_1 \\
        0 & 1
    \end{bmatrix}
    \begin{bmatrix}
        R_2 & T_2 \\
        0 & 1
    \end{bmatrix} = 
    \begin{bmatrix}
        R_1R_2 & R_1T_2+T_1 \\
        0 & 1
    \end{bmatrix}$$
    \item Euklidische Bewegungen sind invertierbar:
    $$M_{-1} = \begin{bmatrix}
        R & T \\
        0 & 1
    \end{bmatrix}^{-1} = 
    \begin{bmatrix}
        R^T & -R^TT \\
        0 & 1
    \end{bmatrix}
    $$
\end{itemize}

\subsubsection{Das Kreuzprodukt}

\paragraph*{Definition}

\begin{itemize}
    \item Das Kreuzprodukt zwischen den Vektoren $\mathbf{u, v}\in\mathbb{R}^3$ ist definiert als
    $$
    \mathbf{u}\times\mathbf{v} = \begin{bmatrix}
        u_1 \\ u_2 \\ u_3
    \end{bmatrix}\times
    \begin{bmatrix}
        v_1 \\ v_2 \\ v_3
    \end{bmatrix}
    \begin{bmatrix}
        u_2v_3-u_3v_2 \\
        u_3v_1-u_1v_3 \\
        u_1v_2-u_2v_1
    \end{bmatrix}\in\mathbb{R}^3
    $$
    \item Es gilt: $<\mathbf{u}\times\mathbf{v},\mathbf{u}> = <\mathbf{u}\times\mathbf{v},\mathbf{v}> = 0$
    und $\mathbf{u}\times\mathbf{v} = -\mathbf{v}\times\mathbf{u}$
\end{itemize}

\begin{figure}[!h]
    \centering
    \includegraphics[width=0.3\textwidth]{../img/2-10.png}
    \label{img/2-10}
\end{figure}

\paragraph*{Matrix-Vektor-Multiplikation}

\begin{itemize}
    \item Das Kreuzprodukt lässt sich schreiben als
    $$
    \mathbf{u}\times\mathbf{v} = \hat{\mathbf{u}}\mathbf{v},\;\text{mit}\;\; \hat{\mathbf{u}} = 
    \begin{bmatrix}
        0 & -u_3 & u_2 \\
        u_3 & 0 & -u_1 \\
        -u_2 & u_1 & 0
    \end{bmatrix}\in \mathbb{R}^{3\times 3}
    $$
    \item Es gilt (für $A$ invertierbar): $\Hat{A\mathbf{v}} = \det (A)A^{-T}\hat{\mathbf{v}}A^{-1}$
\end{itemize}

\paragraph*{Eigenschaften Orthogonaler Transformationen}

\begin{itemize}
    \item Orthogonale Transformationen erhalten Skalarprodukt $<\mathbf{u,v} = <O\mathbf{u},O\mathbf{v}$
    \item Euklidische Transformationen erhalten Abstand
    \item Spezielle orthogonale Transformationen erhalten das Kreuzprodukt $R\mathbf{u}\times R\mathbf{v} = R(\mathbf{u}\times \mathbf{v})$
\end{itemize}

\subsubsection{Zusammenfassung}

\begin{itemize}
    \item Kamerabewegungen durch Koordinatenänderung eines festen Raumpunktes.
    \item Koordinatenänderung durch euklidische Bewegungen.
    \item Rotationen durch spezielle orthogonale Matrizen.
    \item Euklidische Bewegung in homogenen Koordinaten durch Matrix-Vektor-Multiplikation.
\end{itemize}



\end{document}